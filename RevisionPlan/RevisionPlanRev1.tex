\documentclass{article}
\usepackage[utf8]{inputenc}
\usepackage{natbib}
\usepackage{graphicx}
\usepackage{geometry}
\usepackage[T1]{fontenc}
\geometry{ margin=1.3in}
\usepackage{enumitem}

\begin{document}
% \begin{figure}[h!]
% \centering
% \includegraphics[scale=0.4]{logo.jpg}
% \label{fig:logo}
% \end{figure}
% \title{Game Design Capstone}
% \author{Yanting Zhang }
% \date{September 2017}

\begin{titlepage}
    \begin{center}
        \vspace*{1cm}
        \includegraphics[width=0.8\textwidth]{logo.jpg}
        \\
        \textbf{\Large Revision Plan - Revision 1}
        \vspace{0.5cm}
        \textbf{\Large  \\Mech-APEX Zero}
        \vspace{1cm}
        \textbf{\\Tian Guo\\Saim Zahid\\Jonathan Yu\\ Yicheng Chen \\Yanting Zhang }
        \vfill
        \vspace{0.8cm}
        \begin{flushright}
        Professor: Dr. Jacques Carette\\
        Team Name: Chicken and Sausages\\
        Course: Software Engineering 4GP6\\
        Date: March 3, 2018
        \end{flushright}
    \end{center}
\end{titlepage}


\subsection*{Revision Change Log}
\begin{itemize}
	\item Separation from High Concept Doc Rev 1 

\end{itemize}


\subsection*{Introduction}
This document will summarize the Git feedbacks of the game Mech-APEX Zero. We will be analyzing each issue and determine what changes should be implemented in the future. We will also be outlining new features that we will be adding to the next revision of the game.

\section*{Bugs and Suggestions}
This section includes bugs and game changing suggestions made by game testers. Some bugs weren't reproduce able and others were the same but reworded. The bugs have been divided into 8 different subsections below. Within the sub sections they will describe the bug, list the priority on fixing it and the estimated difficulty.Priorities range from high, medium and low. Difficulty ranges from easy, medium and hard.\\

The rational behind the priority range depends on the occurrences of the issue, the amount of risk and the impact on the player's experience. 

All player,  enemy, audio and instructional bugs will be fixed in the next revision, any bug with an easy tag will likely be fixed, in fact many have already been fixed. Any bug tagged with difficult may not be fixed in time if the written solution doesn't work  and can't be solved easily. Many of the visual bugs regarding  graphic overlapping, UI sizes at different resolutions and mapping tiles not perfectly lining up may not be fixed in the next revision.



For instance, the graphics overlapping with walls may have medium occurrences but don't affect the game-play at all, in fact most of the issues that are visual bugs will have a low priority status.  


\subsection*{Visual Bugs}
\begin{enumerate}
		\item Player GE Graphic overlaps with walls (low, medium)
		\item Player appears to be floating due to leg height(low, difficult)
		\item Player death playing multiple times after death (low, easy)
		\item The player looks as though he should fall off a platform but doesn't (low, hard)
		\item If you press "D" "A" "D" or "A" "D" "A" you can dash but have the player's graphic in the wrong  direction. (low, medium)
			\item Main menu  buttons have too much space on high resolution (low, medium)
				\item Background tiles does not tile smoothly (low, easy)
\end{enumerate}
\subsection*{Player Bugs}
\begin{enumerate}
	\item The player while crouching doesn't go into his jumping animation. (high, easy)
		\item Weird hitboxes on your sword and potentially larger sword
			\item Everything is too floaty (high, medium)
				\item It is possible to delay vertical movement almost indefinitely while maintaining horizontal movement (high, easy)
					\item Getting hit by boss's second fire attack while kneeling stun locks the player (high, easy)
						\item kneel walking (High, easy)
							\item Attacking while dashing performs the first few frames of the attack then stops, and only works the first time (after that pressing J while dashing does nothing). (High, medium)
								\item Can't jump on an enemies head (medium, easy)
									\item player can produce walk animation while sprite stays still (High, easy)
									\item Crouching and shooting does not initiate animation, fires projectiles from above player sprite (High, easy)
\end{enumerate}
\subsection*{Enemy Bugs}
\begin{enumerate}
	\item Enemy Attacks walls if the player is above it. (High, easy)
		\item Rolling enemies do damage to player even though the enemy is dead (High, easy)
			\item Boss flame charge  hard to avoid (High, medium)
				\item Enemies attack aren't interrupted by player attacks.  (high, hard)
					\item Boss can get stuck on ledge when jumping (high, medium)
						\item The enemy flickers between walking left and right animations  (High, easy)
							\item Boss hit me through the floors  (High, hard)
								\item Spikes are way too dangerous. (low, easy)
									\item Barely on screen enemies can be seen and run animations but don't move (low, easy)
										\item Spikes don't kill enemies (low, easy)
										
\end{enumerate}
\subsection*{Audio Bugs}
\begin{enumerate}
	\item Background music doesn't loop (medium, easy)
		\item Colliding with moving block/wall/"gate" after it has opened still produces the opening sound effect (low, easy)
			\item Sword sound-effect plays suddenly and repeatedly while in particular location (low, easy)
\end{enumerate}
\subsection*{UI Bugs}
\begin{enumerate}
		\item No pause menu (High, medium)
			\item Can not quit game from main menu (high, easy)
				\item Small resolution causes mini map to take a big percentage of the screen space (low, medium)
					\item Free Mode level select isn't working correctly (low, easy)
				
\end{enumerate}
\subsection*{Instructional Bugs}
\begin{enumerate}
	\item Instructions aren't clear which lead to death, It says "double tap D to dash" but it should say double tap and hold to dash.(low, Easy)
	\item Goal given to the player at the start of the game.  (low, Easy)
		\item  Label the items the players are buying (medium, easy)
			\item Not understanding how I gained currency
				\item Certain moves are unknown to the player  (medium, easy)
\end{enumerate}
\subsection*{Suggestions}
\begin{enumerate}
	\item  Should give the player to pick their own control scheme to play game (low, medium)
		\item No Game Over Screen (low, easy)
			\item Sword is not useful  (High, hard)
				\item Unable to save progress. (medium, Hard)
					\item Suggestion: Allow player to skip cutscenes (low, medium)
				\end{enumerate} 
\subsection*{Resource Bug}
\begin{enumerate}
	\item Already used repair kits re spawns back after death (low, medium)
	\item All resources are lost when dying and returning to a checkpoint  (low, easy)
\end{enumerate}





\section*{Plans for Bug Fixing}
This sections describes the approach and the feasibility for fixing the bugs mention in the bug section.
Each bug above has a number associated with it above. The number on this list is associated with fixing that bug within its section. 
\subsection*{Visual Bugs}
\begin{enumerate}
		\item In order to fix the overlapping of our character graphic and the wall graphics is simple. We can easily increase the size of our box collider to be larger than our player's graphic. The problem with this solution is now our box collider is larger than our player's graphic. We wanted the box collider to be smaller than the player's graphic
		to the player's benefit. Another solution is to have multiple box colliders that will shape the player. We can prolly have a small box collider at the right most of the shield that will prevent the player moving it into wall. I don't think that it would be possible for us however to have a 1 to 1 collision detction with the Gundam Graphic. It would take a considerable amount of work: due to the increase the number of box colliders. We still don't completely understand the ramifications of inserting more box colliders. It could tax the computer more harshly, if we wanted to change the animation state of the character than we would have to make changes to the position and size of each collider therefore increasing the amount of work a lot. Also by changing the size and moving box colliders the affects aren't clear on unity and may introduce many number of bugs. For instance, crouching in our game decreases the size of the player's box collider but that change in size sometimes allows the user to clip through objects when the box collider is changing size.  For the amount of work, there is very little value gained, we have more accurate hit boxes but most times the player doesn't even notice or cares as long as it is too his advantage.The player's collision box was never intended to be very accurate but we will attempt to add one or two to prevent character's overlapping with walls.
		\item this bug is similar to the one above but harder to fix due to leg's at different height in idle animation. There isn't much of a solution here due to the difference in height of the leg. However we will tweak with the box collider and attempt to get it as close to the ground as possible
		\item Set the rigidbody.simulated to false when the player dies 
		\item The solution to this is  tweaking our box collider to more accurately represent the player. Specifically when the player is in his jump animations.
		\item In order to fix there would need to be a tweak on the dash starting conditions.
			\item we shall have main menus scale to the resolution size.
			\item line up our background tiles properly
\end{enumerate}
\subsection*{Player Bugs}
\begin{enumerate}
	\item crouching wasn't in our initial requirements but was added in after playing the game and feeling like it would be a fun mechanic to add in. It's functionality wasn't tested much therefor many bugs will may include in it.In order to fix this bug, in our GE animation controller had a transition from the crouching state the to jump state. 
		\item The sword's hit box is always larger than the sword's graphic. However tweaks to the size and position of the sword's hotbox will be made to more intuitive. The main bug that nobody seemed to address was attacks aren't interrupted on hit which causes a lot of invisible hit boxes. 
			\item In order to fix everything is too floaty. After the player releases the jump button the gravity scale will be increased significantly. This will allow the player to get down on the ground quicker and still be able to jump high and have control in the air.
				\item back dashing  wasn't in our requirement  but it was a really nice addition to our game. That is why there are many bugs including this and crouching. As well as the fact, there were no instructions on them.In order to prevent this bug we won't allow back dashing when the player is in the air.
					\item create a animation transition from crouching to hurt and allow movement from the crouching state.
						\item Set a animation transition from crouching to walking and allow walking from crouching.
							\item The intended behavior is that player can only dash attack once and when he does his movement should stop. To fix the bug, once the player performs a dash attack it should set the velocity of the character to zero and send him back to the idle state. The player should not be able to dash attack in the air.
								\item set the enemy layer mask to "Ground", so know the player will be considered grounded when atop of the enemy, allowing him to jump.1
									\item fixed the same as kneel walking
										\item prevent shooting while crouching through boolean conditions.
\end{enumerate}
\subsection*{Enemy Bugs}
\begin{enumerate}
		\item Find the delta Y of the player and the GE and add it in the preconditions for attacking. deltaY < 5 in order to attack. 
			\item this was intended at first due to the enemy exploding on death but in retrospect it is unfair to the player. There are many ways to fix however our fix is to change the rb2d.simulated to false once the enemy dies, therefore it will no longer damage the player and the exploding animation will finish.
				\item All moves are avoidable the player however I will agree that the charge attack by the boss is hard to dodge. The user would have to jump early expecting the flame charge because he doesn't reach the peak of his jump quickly enough in order to dodge the attack, Also the charge isn't telegraphed enough. Its easy for me the developer to dodge because I know the exact sequence of the boss move. If the player was observant enough he would know the boss cycles between jumping 3 times, than shooting on the ground, than the flame charge but that of course is expecting too much of the player. To fix this we will likely slow down the flame charge attack and include a pre animation of charging and a time delay before the charge attack.
					\item If the enemy calls an attack function the move happens regardless if they are put into hitstun. This is why so many play testers feel that the melee weapon is terrible and is trading hits with his enemy. In order to fix this when the enemy gets hit by the player it will stop all running coroutines. This will prevent the attacks from coming out.
						\item two solutions, One is to increase the ledge height therefore the boss can't get stuck on the ledge. Whats happens  is  his hand hit box gets stuck on the ledge. Likely won't be using this solution. Another solution is we will create a an invisible wall trigger that will act like a wall. When Fire Dino jumps into the trigger his x velocity gets flipped and body graphic gets flipped preventing him from getting stuck on the ledge.
						
						\item The enemy flickers between left and right when the enemy is on top of the player because its position is jittering switching between greater than the players position and less than its position. The fix will be to add the condition delta > 0.5 to both moving left and right. This will create a range where the enemy will not try to switch between moving left and right
							\item The mass of the enemy's boss's rigidy body and it's velocity when it collides with the player sometimes the force it sends the player at is too great and the player goes through the floor. To fix this bug change the player's CollisionDetectionMode to ContinuousDynamic.
								\item Spikes will no longer instantly kill the player and do damage comparable to enemies
								\item in the update function set the enemies animation to idle if it's velocity.x is zero.
									\item Add the tag "Spikes" to the enemies onCollsionEnter2D function and run the function enemyGetHit(spikeDamage);
\end{enumerate}
\subsection*{Audio Bugs}
\begin{enumerate}
		\item Replace playOneShot in the sound manager with the command play.
		
		\item This sound effect is caused by GE because his attack conditions doesn't take into the players y position . In his attack condition add  of deltaY < 5 
		
			\item in OnCollision check if !isOpen is true if it is than play the gate sound.
\end{enumerate}
\subsection*{UI bugs}
\begin{enumerate}
		\item Pause menu will be implemented in this revision. Pressing 'esc' or the pause button in the Unity development kit. It will set the Time.scale to zero and enable a UI game menu.
			\item Write the function endGame(){Application.QuitGame()} Attach that function onClick() to the exit button.
				\item we shall have mini map scale to the resolution size.
					\item Not all levels implemented yet, When they are finished they will send you to the correct level.
\end{enumerate}
\subsection*{Instructional Bugs}
\begin{enumerate}
	\item Change text to "double tap and hold D to dash". Since we are now using the Unity development kit for controls the text will have to be dynamic so we have use the keytag.ToString().
	 \item It isn't necessary to tell the player the goal of the game at the start of the game. The player will learn about his goals while playing through the level. However we will be adding in coins to the game, that give the player scrap, but is actually a path that will guide the player to his destination and be a sub goal.
	 	\item The names are already within in the games code just haven't created text to display it. Add in text to be displayed. However the names alone won't exactly tell the player what the item does. Description will likely be added but how that is displayed to the player is the hard part. The simplest answer is just to have a paragraph that pop ups above the player, however that will be tweaked on play test.
	 		\item scrap is currently gained when the player destroys an enemy. An observant player would notice that easily and I don't think it is necessary to tell the player where he got his currency from. He will naturally know after playing and this information isn't important to progressing through the game. 
	 			\item Instruction will now be added for crouching and back dashing 
\end{enumerate}
\subsection*{Suggestions}
\begin{enumerate}
	\item We will now be using Unity's tools for controls simple due to the high demand for changing controls and it isn't hard to change. The main benefit however is going through the main menu using the control keys.
		\item The player gets re spawned to his last checkpoint when he dies. I don't think it is necessary to have a game over screen. If we did want a game over system, we would likely introduce a life system and when it reaches zero we go to the game over screen. We don't want this because we want our players to actually finish our game and not be too frustrated from having to replay areas of the game in order to get back to where they started. 
			\item Sword in-game will now interrupt GE attacks and do more damage
				\item A new feature will be implemented allowing the user to save their game
					\item The scene has 3 dialog texts, there is no reason to include a skip on the dialog.
\end{enumerate}
\subsection*{Resource Bugs}
\begin{enumerate}
		\item This was intended as we were attempting to make the game as easy and forgiving to the player as possible. 
			\item When the player hits new game from the main menu it should reset the player's scrap and repair kits and not when the player enters the level. 
		
\end{enumerate}


\section*{Bug Impact}
This section will explain the impact the bug has on the player's experience, risk and likely hood of occurrences. Each bug in the "Bug and suggestions" above has a number associated with it above. The number on this list is associated with fixing that bug within its section.  

\begin{enumerate}
	\item The occurrence of this is quite likely but it doesn't affect the game play experience. Many games have graphic overlap and the player generally assumes that the wall is in the background.
	\item The occurrence of this is likely but it doesn't affect the game play. The player is only slightly hovering due to the right leg being slightly higher. 
	\item The occurrence of this isn't that likely the player has to be moving into a hurtbox while dieing. It doesn't affect game play at all but may break his immersion and gives the game a buggy feeling.
	\item  The occurrence of this is medium but it doesn't affect game play much but does change the feel of their character
	\item The occurrence of this is unlikely, the inputs for this are very precise and not common. The player may be confused by their graphic dashing the wrong way.
	\item This affect the polish of the game
	\item Doesn't affect game play but affects the player's environmental experience
\end{enumerate}
\subsection*{Player Bugs}
\begin{enumerate}
	\item This is a bad bug and will be fixed immediately. Confuses the player and is likely to occur
	\item this affects the player's experience, he may feel that the sword is useless and never use it.
	\item This affects the player's pace and experience, it also limits our design space. It can prevent the player from dodging moves and make the game feel very slow because there are less options in the air.
	\item This is a game breaking bug and will be fixed immediately, the player should not be able to hover
	\item This is a game breaking bug, the player should not be stuck in crouching position. 
	\item This is a bad bug, the player should not be able to move while in crouching animation.
	\item This doesn't affect game play much however melee attacks in last revision weren't very useful and we would like to change that. 
	\item This doesn't change the game much.
	\item This is a bad bug, the player should not be able in walk animation but not move.
	\item This is a visual bug which will be fixed due to being confusing to the player.
\end{enumerate}
\subsection*{Enemy Bugs}
\begin{enumerate}
	\item this bug doesn't change game play much but it does make the enemies more believable 
	\item this bug makes the game harder and more frustrating for the player, further weakening the sword ability.
	\item this bug makes the game harder,unfair and more frustrating for the player.
	\item this bug makes the game harder and more frustrating for the player, further weakening the sword ability.
	\item The boss getting stuck on the ledge gives the game a buggy feel and makes the boss fight easier.
	\item The enemy flickers between walking left and right animations makes the game feel buggy but doesn't change game play much. 
	\item Boss hit me through the floors. This bug is game breaking and can potentially soft lock the game.
	\item Spikes are way too dangerous. This isn't a bug but can be very frustrating for player's to die in one hit.
	\item This is a visual bug, it doesn't affect gameplay but gives the game a buggy feel. 
	\item This makes it feel like the environmental hazards are on the enemies side which feels very unfair to the player.
	
\end{enumerate}
\subsection*{Audio Bugs}
\begin{enumerate}
	\item If the player manages to finish the level before the first loop ends, no harm is done. This changes the feel of the level after the music ends.  
	\item This bug doesn't affect game play, the player constantly jumping into the door is unlikely and the replay of the audio doesn't do much.
	\item This bug is very annoying to the player and can confuse him.
\end{enumerate}
\subsection*{UI Bugs}
\begin{enumerate}
	\item If the player manages to get the player stuck or needs to go to the washroom. He can't pause.
	\item Can not quit game from main menu. This bug makes the game feel buggy if the exit button isn't working.
	\item It is uncommon for people to play games at 640 x 480 resolutions and screen size is still feasible to play.
	\item This bug can confuse the player
\end{enumerate}
\subsection*{Instructional Bugs}
\begin{enumerate}
	\item Can cause the player to walk into their death but they are re spawn very close to the death spot and can try again. 
	\item The player doesn't to be told the goal of the game, if the level is well designed.
	\item  The player can be very confused on exactly what the items do and may not want to spend their scrap on item's without description.
	\item Not understanding how I gained currency. It doesn't matter whether or not the player knows how they got their currency. 
	\item The other moves aren't necessary for the player to complete the level and were work in progress abilities. 
\end{enumerate}
\subsection*{Suggestions}
\begin{enumerate}
	\item  It can confuse the player when they are given the ability to change their controls on the unity start up but then those controls don't work in game. 
	\item There doesn't need to be a game over screen, it makes the game a lot easier and doesn't make you redo certain parts of the level.
	\item this affects the overall feel of the game, A game about a combo's shouldn't have the gun as the dominant strategy.
	\item The game currently is very short so saving progress at the moment is very necessary and doesn't impact the player much 
	\item the dialog is 3 lines which is done
\end{enumerate} 
\subsection*{Resource Bug}
\begin{enumerate}
	\item This makes the game easier by having those resources re spawn on death.
	\item This makes the game harder by losing all resources on death.
\end{enumerate}

\section*{Plans for New Features for Functional and Non-Functional Requirements} 
This section describes the functional/nonfunctional requirement of new features that will be implemented within the game. This section also explains the rational for adding them and how they fit within the game.
\subsection*{Features}
\subsubsection*{Air Combo Attacks}
Currently within the game there is already a feature that allows the player to knock the enemies within the air, the player can currently only do one air attack after the ground launcher.  We would like to extend this functionality allowing the player to attack multiple times within the air. This will make the game more fun

\subsubsection*{Level Fades}
Scenes must fade to black upon exiting a scene and scenes must fade out of block upon entering a new scene. This is to allow a smooth transition between other scenes and to prevent a jarring switch between two scenes.

\subsubsection*{Save Progression}
The player will be allowed to save and load their game. Saving will store information regarding the player's scrap, repair kits, checkpoint, the player's level, his health, his attack, his defense stat and energy. This will likely be done by saving to a json file. This is so that the player does not have to complete the game in one sitting.

\subsubsection*{Player Re spawn Introduction}
 When starting a level, no longer will the Gundam just be there and the player has immediate control. There will be a spawn sequence that includes GW now being carried on a base jabber which will spawn from the left of the screen and than GW will than jump off and land on the re spawn plate. This is so that the game feels more natural 

\subsubsection*{Damage Display}
When an enemy or player gets damaged their will be a red text that will pop up to the corner left of the player. That text will be the amount of damage that it has taken. The text will than fade away after a second. This allows the player to see the exact damage he is doing which give the game a nice polish.

\subsubsection*{Combo Text Display}
Text will be displayed on the corner right of the player when the player ends or drops his combo. The text displayed will be the number of hits in their combo. This is so that the player knows what moves are combing and can check what his longest combo was.  

\subsubsection*{Enemy Tweaks}
The major complaint of GE is that his attack was deemed unfair. He would attack too fast and it felt like the player was his trading life for his. Our solution will be to have the main enemy go into a pre attack animation state for 0.7f seconds. It will than remember the last position of the player. It will then move toward that remembered position after the elapsed time and it's animation will go to the end of the attack. This is to make the enemies more fair and the game more fun.

\subsubsection*{Level 2}
In this revision our main aim to create a new level, that includes a well designed stage and a brand new boss. The level will be thematically designed based on the boss's character and personality. It will be his fortress. This will increase the length of the game and give player's new challenges and experiences. 

\section*{New Feature Design Plans}
\subsection{Air Combo Attacks}
The image below displays the sequence of attacks the player can make in the air. The last hit will knock the enemy away. We will create an Array of Attack[] called airAttacks which will contain our sequence of air attacks. We shall than pass that array into our combo attack function.IEnumerator comboAttacks(Attack[] attacks). Refer to player controller


       \includegraphics[width=0.8\textwidth]{airCombo.png} \\
        Below  is the planned attack data for each air attack \\
        
        airAttack1  
        \begin{enumerate}[noitemsep,nolistsep] 	 
        	\item attackDamage = 3
        	\item attackCollider = airAttack1 
        	\item startDelayTime  = 0.3f
        	\item durationTime = 0.5f
        	\item knockBackForce = zero
        	\item hitStunTime = 1
        	\item animationTrigger = "AirAttack1"
        	\item attackSound = "Air Swing"
        	\item cancelTime = 0.3f
        \end{enumerate} 
                airAttack2  
                \begin{enumerate}[noitemsep,nolistsep] 	 
                	\item attackDamage = 2
                	\item attackCollider = airAttack2 
                	\item startDelayTime  = 0.3f
                	\item durationTime = 0.6f
                	\item knockBackForce = zero
                	\item hitStunTime = 1
                	\item animationTrigger = "AirAttack2"
                	\item attackSound = "Air Swing2"
                	\item cancelTime = 0.3f
                \end{enumerate} 
        airAttack3 
         \begin{enumerate}[noitemsep,nolistsep] 	 
        	\item attackDamage = 5
        	\item attackCollider = airAttack3 
        	\item startDelayTime  = 0.3f
        	\item durationTime = 0.5f
        	\item knockBackForce = 300f
        	\item hitStunTime = 1
        	\item animationTrigger = "AirAttack3"
        	\item attackSound = "Air Swing3"
        	\item cancelTime = 0.3f
        	\end{enumerate} 
\subsubsection*{Level Fades}
there will be Texture2D called the fadeOutTexture which will be a black.png that will cover the player's camer. There will be a function that begins fade which will accept an int for direction. 1 being fade in and -1 being fade out. Begin fade will slowly increase or decrease the opacity of the black texture.
\subsubsection*{Save Progression}
There will be a save block that the user hits in order to save. There will always be one in the Boss walkway 
\subsubsection*{Player Re spawn Introduction}
The base jabber will be instantiated off the camera than enter the scene from the left side and than out through the ride side. When the base jabber x positions is above the re spawn point the GW will jump off the base jabber and land on the platform once the player has landed on the platform. The player gains control over GW.



        \includegraphics[width=0.8\textwidth]{baseJabber.png} \\
        the player on a base jabber



\subsubsection*{Enemy Tweaks}
Enemy GE will have a preAttack animation state and a end attack animation state. There transitions will be through triggers.


\includegraphics[width=0.8\textwidth]{preAnimation.png}

\subsubsection*{New Level}

Here is a sketch of the intended level for level 2 \\
\includegraphics[width=0.8\textwidth]{level2.png} \\

\subsection*{Project Timeline}
This section only contains one chart for the project time line. Please refer to ProjectTime.xlsx for more charts, the complete timeline and responsibilities. \\
\includegraphics[width=0.8\textwidth]{ProjectTime.pdf} \\
\end{document}
